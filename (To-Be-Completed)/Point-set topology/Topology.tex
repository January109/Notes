\documentclass[11pt]{article}
\usepackage{styles}
\title{Point-set topology notes}
\begin{document}
\maketitle
These notes are mainly my own rendition of roughly how I would teach or think about point-set topology, written in a way to give appreciation to its basic (and usually perceived to be dry) notions. 
\tableofcontents
\np
\section{Motivation}
We describe some of the shortcomings with abstract metric spaces, as well as how the notion is usually adapted while still keeping some of the important underlying structure.
\subsection{Lack of universal morphisms}
In abstract metric spaces, (from a more algebraic perspective) there isn't a universally accepted notion of a morphism or ``structure preserving map'' $f : X \to Y$ between two metric spaces $X, Y$.

We have \emph{isometries} which preserve the exact distance (i.e. by setting $d_Y(f(x_1), f(x_2)) = d_X(x_1, x_2)$). This notion is ``too strong'', as the map $f$ is then necessarily injective (forcing all such maps to necessarily be inclusions), and the image $f(X) \subseteq Y$ can be viewed equivalently as an exact ``copy'' (from a metric space perspective) of $X$ in $Y$.

Weakening slightly, we have \emph{Lipschitz continuous} maps, which are such that $d_Y(f(x_1), f(x_2)) \leq Cd_X(x_1, x_2)$ for some fixed constant $C$. This notion of a ``structure preserving map'' is less restrictive and allows for non-injective maps, and so is slightly more ``sensible'' than an isometry from this perspective. The associated ``isomorphisms'' would then be the maps for which the metrics are comparable or equivalent, and so these maps are likely the most sensible morphism notion. Lipschitz continuous maps preserve most notions, but as with the remaining structure preserving maps, they do not preserve the exact distances (which as above is too strong for such a notion).

We have uniform and normal continuous maps as notions of a structure preserving map. Here, \emph{continuity} of a map $f : X \to Y$ is usually phrased as saying that ``for any $\e > 0$ and any $x$ in $X$, we can find $\delta > 0$ with $d_Y(f(x), f(y)) < \e$ for any $y$ with $d_X(x, y) < \delta$'', and \emph{uniform continuity} is the same statement where the same value of $\delta$ works for all pairs of $x$ and $y$ whose images under $f$ are at most $\e$ apart. 

Continuity and uniform continuity are inherently local properties, as the $\e$-$\delta$ definitions are inherently skewed towards smaller values, in the sense that if $0 < \e_1 < \e_2$, then a value of $\delta$ satisfying the (uniform) continuity constraint for $\e_1$ will also work for $\e_2$. In particular, these are local properties, and often fail to capture the geometry induced by the metric structure. For example, the identity function is a uniform equivalence (and hence homeomorphism) between $\Z$ equipped with the discrete and Euclidean metrics respectively. However, the distances associated to $\Z$ with the discrete metric would make it the analogue of a triangle with countably infinitely points, while the Euclidean metric would make it a line.

% Uniform continuity; continuity.
\subsection{Metric spaces can only ``be so big''}
In a metric space, every pair of points necessarily have only a finite distance between them (so in principle you could walk from any point to another). Explicitly, a metric space structure on a set $X$ involves a function $d : X \times X \to \R_{\geq 0}$, explicitly making a metric space ``locally as big as $\R$''. This prevents metric spaces from being too pathological, but also forces any metric space with interesting structure to be relatively ``small''. We describe some sensible spaces without suitable metrics for the associated geometry.
\begin{example}[Non-metric spaces]\label{nonmetric}
    \begin{enumerate}
        \item (Functions) Let $(a, b) \subseteq \R$, $k \in \N_0$ and consider $C^k((a, b), \R)$. For any $[a', b'] \subseteq (a, b)$, $C^k([a, b], \R)$ is a metric space with the standard $\ell^p$ ($p \geq 1$) norm
        $$
            \norm{f}_p = \brac{\int_{a'}^{b'}\abs{f(x)}^p dx}^{1/p}
        $$
        as such functions are necessarily bounded since $[a', b']$ is compact in $\R$. This metric cannot be extended naturally to $C^k((a, b), \R)$ since the boundedness of a function $f : (a, b) \to \R$ is not guaranteed, and in fact we can have functions which grow arbitrarily fast near either $a$ or $b$, such as $f(x) = (x - a)^{-1/p}$. 
        
        Since every $C^k$ function on $(a, b)$ restricts to a $C^k$ function on any $[a', b'] \subseteq (a, b)$, $C^k((a, b), \R)$ can be viewed in some part as the ``limit'' of the spaces $C^k([a', b'], \R)$, under which this failure to yield a metric space can be viewed as metric spaces not being closed under these limiting procedures.
        \item (Sequences in $\R$) The space $\R^\infty$ of real-valued sequences (viewed as infinite vectors indexed by $\N$) naturally contains a copy of $\R^n$ for each $n \in \N$, by taking the first $n$ coordinates of each sequence. Each $\R^n$ is usually equipped with the Euclidean metric, and so $\R^\infty$ with the Euclidean metric (so that the inclusions $\R^n \hookrightarrow \R^\infty$ are isometries) is not an unreasonable notion to consider. This is however too much to ask for, and the best which can be done in the world of metric spaces is $\ell^2$, defined to be exactly those sequences which have finite norm (that is, are a finite distance from $0$).
        \item (The long line) One way to view $\R$ is as the disjoint union
        $$
            \R = \coprod_{n \in \Z}[n, n + 1)
        $$
        From this perspective $\R$ can be viewed as $\abs{\Z}$ copies of the interval $[0, 1)$ patched together at the endpoints, i.e. ``$\R \cong \Z \times [0, 1)$''. Replacing $\Z$ with a well-ordered set of bigger cardinality\footnote{This step explicitly requires the well-ordering principle, equivalent to the axiom of choice}, we can find longer line-like spaces, of which a metric cannot capture the true geometry.
    \end{enumerate}
\end{example}
\subsection{Infinitude of products}
The failure of $\R^\infty$ to be a metric space in spite of $\R^n$ (for $n \geq 1$) is an example of a more general failure of metric spaces to generalise to infinite products. For a finite product 
$$
    X = \prod_{i = 1}^k X_i
$$
of metric spaces $X_i$ with associated metrics $d_i$, we have standard $\ell^p$ product metrics of the form
$$
    d^p((x_1, \ldots, x_k), (y_1, \ldots, y_k)) = \brac{\sum_{i = 1}^kd_i(x_i, x_j)^p}^{1/p}
$$
for which the canonical inclusions $X_i \hookrightarrow X$ are isometries, but as we saw in the above example, even when the index set $\Lambda$ is just countable, $$
    X = \prod_{\a \in \Lambda} X_\a
$$
may not be a metric space, let alone for even larger index sets. In this sense, metric spaces are not closed under arbitrary products. To allow for a suitable notion which is closed under products, we need a more general notion independent of $\R$.
\section{Topological spaces}
Unless otherwise stated, $X, Y$ and $Z$ are topological spaces, and $\Tau_S$ (or sometimes simply $\Tau$) is the set of open sets of $S$.
\subsection{Basic definitions}
Leaving only the notion of an open set, which we can think of in the same way as a metric space (i.e. a set in which every point is still locally ``far enough'' from its edge). We abstract the properties of open sets from metric spaces, yielding the following notion.
\begin{definition}[Topological space]
    Let $X$ be a set. A topology $\Tau$ on $X$ is a set of subsets $\Tau \subseteq \P(X)$ satisfying
    \begin{enum}
        \item $\emptyset, X \in \Tau$;
        \item if $\set{\U_\a}_{\a \in \Lambda}$ is a collection of sets in $\Tau$, then $\bigcup_{\a \in \Lambda} \U_\a \in \Tau$; and
        \item if $\set{\U_i}_{i = 1}^n$ is a connection of sets in $\Tau$, then $\bigcap_{i = 1}^n \U_i \in \Tau$.
    \end{enum}
    The sets in $\Tau$ are referred to as \emph{open sets}, and we say that a space $X$ equipped with a topology $\Tau$ is a \emph{topological space}.
\end{definition}
The notion of a topological space then naturally generalises the notion of a metric space, in the sense that any metric space is a topological space with its usual open sets. Since topological spaces are a vast generalisation of metric spaces, it is usually not hard to find cathartic or pathological examples of such spaces.

In a metric space, to talk about the space near a given point $x$, we usually refer to the open balls $B_r(x)$. Since topological spaces are no longer tied to $\R$ (and hence have no notion of distance), we instead have the following analogue.
\begin{definition}[Neighbourhood]
    Let $x \in X$. A \emph{neighbourhood} of $x$ is an open set $\U$ containing $x$.
\end{definition}
As a natural generalisation of metric spaces, it will also be convenient to talk about closed sets, which we can view in the same light.
\begin{definition}[Closed set]
    We say that $C \subseteq X$ is \emph{closed} if $C^c$ is open.
\end{definition}
The above definition is relatively opaque, and appealing to our metric space intuition, we would expect a closed set to ``contain its boundary points''. To make this precise, we describe the following types of points.
\begin{definition}[Limit point; isolated point]
    Let $A \subseteq X$ We say that a point $x \in X$ is
    \begin{enum}
        \item a \emph{limit point of $A$} if every neighbourhood $\U \ni x$ has non-empty intersection with $A \setminus \set{x}$, that is 
        $$
            (A \setminus \set{x}) \cap \U \neq \emptyset
        $$
        \item an \emph{isolated point of $A$} and there is a neighbourhood $\U \ni x$ which does not meet $A \setminus \set{x}$.
    \end{enum}
\end{definition}
The limit points of a set are thus those for which ``every neighbourhood meets the set'', or for which other points in the set ``approximate the point well''. The constraint $(A \setminus \set{x}) \cap \U \neq \emptyset$ can instead be written $A \cap \U \supsetneq \set{x}$. The isolated points are then in some sense an opposite, in that they are points in the set ``isolated from the remainder of the set'', or points not ``well-approximated by other points in the set''.

An intuitive notion of boundary points of $A \subseteq X$ will be encapsulated in part by our notion of limit points. Keeping with our intuition that closed sets are those which contain their boundaries, we may expect the following result.
\begin{proposition}
    A subset $A \subseteq X$ is closed if and only if it contains all its limit points.
\end{proposition}
\begin{proof}
    $\implies$: Let $A$ be closed. Then $A^c$ is a neighbourhood of any $x \not\in A$ which does not meet $A$, and so as any limit point $x$ of $A$ cannot have such a neighbourhood, we must have $x \in A$.

    $\impliedby$: Suppose $A$ contains all its limit points. Then every $x \not\in A$ has a neighbourhood $x \in \U_x \subseteq A^c$, and so taking the union of all such $\U_x$, we find that
    $$
        A^c = \bigcup_{x \not\in A}\U_x
    $$
    is open, so $A$ is closed.
\end{proof}
% Open sets are ``large'' and closed sets are ``small''

% In this case, for any subset $A \subseteq X$, we can talk relatively about the smallest open set contained in (i.e. the ``smallest large set'' relative), and the largest closed set containing (i.e. the ``largest small set'' relative).

\subsection{Continuous maps}

% Limit points, isolated points

% Continuity from a metric and algebraic perspective
% Classical metric space formulation

% ``Arrows'' i.e. morphisms
% Pullbacks of open sets, compatibilities of topologies
% Homeomorphisms specifying that topological properties are fully preserved

% Subspaces, products, quotients (universal properties)

% Bases, subbases

% Sequences and related discussion
%   In abstract metric space theory, sequences play an important role and are on the same level as many non-sequence constructions.

\end{document}