\documentclass[11pt]{article}
\usepackage{styles}
\title{Point-set topology notes}
\begin{document}
\maketitle
These notes are mainly my own rendition of roughly how I would teach or think about point-set topology, written in a way to give appreciation to its basic (and usually perceived to be dry) notions. 
\tableofcontents
\np
\section{Motivation}
We describe some of the shortcomings with abstract metric spaces, as well as how the notion is usually adapted while still keeping some of the important underlying structure.
\subsection{Lack of universal morphisms}
In abstract metric spaces, (from a more algebraic perspective) there isn't a universally accepted notion of a morphism or ``structure preserving map'' $f : X \to Y$ between two metric spaces $X, Y$.

We have \emph{isometries} which preserve the exact distance (i.e. by setting $d_Y(f(x_1), f(x_2)) = d_X(x_1, x_2)$). This notion is ``too strong'', as the map $f$ is then necessarily injective (forcing all such maps to necessarily be inclusions), and the image $f(X) \subseteq Y$ can be viewed equivalently as an exact ``copy'' (from a metric space perspective) of $X$ in $Y$.

Weakening slightly, we have \emph{Lipschitz continuous} maps, which are such that $d_Y(f(x_1), f(x_2)) \leq Cd_X(x_1, x_2)$ for some fixed constant $C$. This notion of a ``structure preserving map'' is less restrictive and allows for non-injective maps, and so is slightly more ``sensible'' than an isometry from this perspective. The associated ``isomorphisms'' would then be the maps for which the metrics are comparable or equivalent, and so these maps are likely the most sensible morphism notion. Lipschitz continuous maps preserve most notions, but as with the remaining structure preserving maps, they do not preserve the exact distances (which as above is too strong for such a notion).

We have uniform and normal continuous maps as notions of a structure preserving map. Here, \emph{continuity} of a map $f : X \to Y$ is usually phrased as saying that ``for any $\e > 0$ and any $x$ in $X$, we can find $\delta > 0$ with $d_Y(f(x), f(y)) < \e$ for any $y$ with $d_X(x, y) < \delta$'', and \emph{uniform continuity} is the same statement where the same value of $\delta$ works for all pairs of $x$ and $y$ whose images under $f$ are at most $\e$ apart. 

Continuity and uniform continuity are inherently local properties, as the $\e$-$\delta$ definitions are inherently skewed towards smaller values, in the sense that if $0 < \e_1 < \e_2$, then a value of $\delta$ satisfying the (uniform) continuity constraint for $\e_1$ will also work for $\e_2$. In particular, these are local properties, and often fail to capture the geometry induced by the metric structure. For example, the identity function is a uniform equivalence (and hence homeomorphism) between $\Z$ equipped with the discrete and Euclidean metrics respectively. However, the distances associated to $\Z$ with the discrete metric would make it the analogue of a triangle with countably infinitely points, while the Euclidean metric would make it a line.

% Uniform continuity; continuity.
\subsection{Metric spaces can only ``be so big''}
In a metric space, every pair of points necessarily have only a finite distance between them (so in principle you could walk from any point to another). Explicitly, a metric space structure on a set $X$ involves a function $d : X \times X \to \R_{\geq 0}$, explicitly making a metric space ``locally as big as $\R$''. This prevents metric spaces from being too pathological, but also forces any metric space with interesting structure to be relatively ``small''. We describe some sensible spaces without suitable metrics for the associated geometry.
\begin{example}[Non-metric spaces]\label{nonmetric}
    \begin{enumerate}
        \item (Functions) Let $(a, b) \subseteq \R$, $k \in \N_0$ and consider $C^k((a, b), \R)$. For any $[a', b'] \subseteq (a, b)$, $C^k([a, b], \R)$ is a metric space with the standard $\ell^p$ ($p \geq 1$) norm
        $$
            \norm{f}_p = \brac{\int_{a'}^{b'}\abs{f(x)}^p dx}^{1/p}
        $$
        as such functions are necessarily bounded since $[a', b']$ is compact in $\R$. This metric cannot be extended naturally to $C^k((a, b), \R)$ since the boundedness of a function $f : (a, b) \to \R$ is not guaranteed, and in fact we can have functions which grow arbitrarily fast near either $a$ or $b$, such as $f(x) = (x - a)^{-1/p}$. 
        
        Since every $C^k$ function on $(a, b)$ restricts to a $C^k$ function on any $[a', b'] \subseteq (a, b)$, $C^k((a, b), \R)$ can be viewed in some part as the ``limit'' of the spaces $C^k([a', b'], \R)$, under which this failure to yield a metric space can be viewed as metric spaces not being closed under these limiting procedures.
        % Update this to the inner product construction.
        \item (Sequences in $\R$) The space $\R^\infty$ of real-valued sequences (viewed as infinite vectors indexed by $\N$) naturally contains a copy of $\R^n$ for each $n \in \N$, by taking the first $n$ coordinates of each sequence. Each $\R^n$ is usually equipped with the Euclidean metric, and so $\R^\infty$ with the Euclidean metric (so that the inclusions $\R^n \hookrightarrow \R^\infty$ are isometries) is not an unreasonable notion to consider. 
        
        We can actually achieve this, but in a rather unsatisfying fashion involving the axiom of choice: beginning with the standard basis vectors $E = \set{e_i \mid i \in \N}$, by Zorn's lemma (equivalent to AC) we can extend $E$ to a basis $\mathcal{E}$ of $\R^\infty$. Define an inner product on $\R^\infty$ by declaring $\mathcal{E}$ to be orthonormal. This induces a norm and hence a metric, and the inclusions are indeed isometries. This construction is however effectively useless from a metric space perspective - the use of the axiom of choice means that we cannot write down $\mathcal{E}$, and actually computing distances between most vectors\footnote{i.e. those with infinitely non-zero coefficients when expressed as an arbitrary formal linear combination of $e_i$} is not feasible. The closest commonly used construction is $\ell^2$, which is the set of $x = (x_n)_{n \in \N} \in \R^\infty$ for which the ``norm''
        $$
            \norm{x}_{\ell^2} = \brac{\sum_{k = 1}^\infty x_n^2}^{1/2}
        $$
        makes sense.
        \item (The long line) One way to view $\R$ is as the disjoint union
        $$
            \R = \coprod_{n \in \Z}[n, n + 1)
        $$
        From this perspective $\R$ can be viewed as $\abs{\Z}$ copies of the interval $[0, 1)$ patched together at the endpoints, i.e. ``$\R \cong \Z \times [0, 1)$''. Replacing $\Z$ with a well-ordered set of bigger cardinality\footnote{This step explicitly requires the well-ordering principle, equivalent to the axiom of choice}, we can find longer line-like spaces, of which a metric cannot capture the true geometry.
    \end{enumerate}
\end{example}
\subsection{Infinitude of products}
The failure of $\R^\infty$ to be a metric space in spite of $\R^n$ (for $n \geq 1$) is an example of a more general failure of metric spaces to generalise to infinite products. For a finite product 
$$
    X = \prod_{i = 1}^k X_i
$$
of metric spaces $X_i$ with associated metrics $d_i$, we have standard $\ell^p$ product metrics of the form
$$
    d^p((x_1, \ldots, x_k), (y_1, \ldots, y_k)) = \brac{\sum_{i = 1}^kd_i(x_i, x_j)^p}^{1/p}
$$
for which the canonical inclusions $X_i \hookrightarrow X$ are isometries, but as we saw in the above example, even when the index set $\Lambda$ is just countable, $$
    X = \prod_{\a \in \Lambda} X_\a
$$
may not be a metric space, let alone for even larger index sets. In this sense, metric spaces are not closed under arbitrary products. To allow for a suitable notion which is closed under products, we need a more general notion independent of $\R$.
\section{Topological spaces}
Unless otherwise stated, $X, Y$ and $Z$ are topological spaces, and $\Tau_S$ (or sometimes simply $\Tau$) is the set of open sets of $S$. The definitions we give in this section are ``very loose'', and it is not difficult to find counterintuitive examples. In any case, we try to paint a certain picture with each definition and avoid counterintuitive examples, to follow a roughly intuitive and sensible progression.
\subsection{Basic definitions}
Leaving only the notion of an open set, which we can think of in the same way as a metric space (i.e. a set in which every point is still locally ``far enough'' from its edge). We abstract the properties of open sets from metric spaces, yielding the following notion.
\begin{definition}[Topological space]
    Let $X$ be a set. A topology $\Tau$ on $X$ is a set of subsets $\Tau \subseteq \P(X)$ satisfying
    \begin{enum}
        \item $\emptyset, X \in \Tau$;
        \item if $\set{\U_\a}_{\a \in \Lambda}$ is a collection of sets in $\Tau$, then $\bigcup_{\a \in \Lambda} \U_\a \in \Tau$; and
        \item if $\set{\U_i}_{i = 1}^n$ is a connection of sets in $\Tau$, then $\bigcap_{i = 1}^n \U_i \in \Tau$.
    \end{enum}
    The sets in $\Tau$ are referred to as \emph{open sets}, and we say that a space $X$ equipped with a topology $\Tau$ is a \emph{topological space}.
\end{definition}
The notion of a topological space then naturally generalises the notion of a metric space, in the sense that any metric space is a topological space with its usual open sets. Since topological spaces are a vast generalisation of metric spaces, it is usually not hard to find cathartic or pathological examples of such spaces, though we will mostly avoid unintuitive counterexamples to make these concepts as illuminating as possible.
\begin{example}[Topological spaces]
    \begin{enum2}
        \item If $X$ is a metric space, the open sets (in the metric sense) form a topology, and this is called the \emph{metric topology} on $X$.
        \item If $X$ is a set, we have two ``simple topologies'':
        \begin{enum2}
            \item The discrete topology is given by $\Tau = \P(X)$. This corresponds to the metric topology where $X$ is given the discrete metric.
            \item The trivial topology is given by $\Tau = \set{\emptyset, X}$. This can be thought of as treating all points in $X$ as ``close enough to be inseparable'', and when $X$ is non-empty, this is equivalent to a single-point topological space in some sense.
        \end{enum2}
    \end{enum2}
\end{example}

In a metric space, to talk about the space near a given point $x$, we usually refer to the open balls $B_r(x)$. Since topological spaces are no longer tied to $\R$ (and hence have no notion of distance), we instead have the following analogue.
\begin{definition}[Neighbourhood]
    Let $x \in X$. A \emph{neighbourhood} of $x$ is an open set $\U$ containing $x$.
\end{definition}
As a natural generalisation of metric spaces, it will also be convenient to talk about closed sets, which we can view in the same light.
\begin{definition}[Closed set]
    We say that $C \subseteq X$ is \emph{closed} if $X \setminus C = C^c$ is open.
\end{definition}
% Dual to definition of open sets

The above definition is relatively opaque, and appealing to our metric space intuition, we would expect a closed set to ``contain its boundary points''. To make this precise, we describe the following types of points.
% Limit points, isolated points
\begin{definition}[Limit point; isolated point]
    Let $A \subseteq X$ We say that a point $x \in X$ is
    \begin{enum}
        \item a \emph{limit point of $A$} if every neighbourhood $\U \ni x$ has non-empty intersection with $A \setminus \set{x}$, that is 
        $$
            (A \setminus \set{x}) \cap \U \neq \emptyset
        $$
        \item an \emph{isolated point of $A$} and there is a neighbourhood $\U \ni x$ which does not meet $A \setminus \set{x}$.
    \end{enum}
\end{definition}
The limit points of a set are thus those which are ``close to the set'', captured by the notion that every open set ``nearby'' also touches the set. We can also think of these as the points which are ``approximated well'' by other points in the set. The constraint $(A \setminus \set{x}) \cap \U \neq \emptyset$ can instead be written $A \cap \U \supsetneq \set{x}$. The isolated points are then in some sense an opposite, in that they are points in the set ``far from other points'' or ``isolated from the rest of the set'', or points not ``well-approximated by other points in the set''.

An intuitive notion of boundary points of $A \subseteq X$ will be encapsulated in part by our notion of limit points. Keeping with our intuition that closed sets are those which contain their boundaries, we may expect the following result.
\begin{proposition}
    A subset $A \subseteq X$ is closed if and only if it contains all its limit points.
\end{proposition}
\begin{proof}
    $\implies$: Let $A$ be closed. Then $A^c$ is a neighbourhood of any $x \not\in A$ which does not meet $A$, and so as any limit point $x$ of $A$ cannot have such a neighbourhood, we must have $x \in A$.

    $\impliedby$: Suppose $A$ contains all its limit points. Then every $x \not\in A$ has a neighbourhood $x \in \U_x \subseteq A^c$, and so taking the union of all such $\U_x$, we find that
    $$
        A^c = \bigcup_{x \not\in A}\U_x
    $$
    is open, so $A$ is closed.
\end{proof}
We can generally think of the open sets as ``large'', as the sets which are ``unbounded'' when viewed from internally. On the flip side, most closed sets can generally be viewed as ``small'', or for which we could ``reach a boundary''. This way of thinking of closed sets however breaks down when talking about sets without boundaries, such as $\emptyset$ or the whole space $X$.

Continuing with this perspective, for any subset $A \subseteq X$, we can talk relatively about the smallest open set inside and outside (i.e. the ``smallest large set'' in the set and its complement), and the largest closed set containing (i.e. the ``largest small set'' containing the set). We can further talk more formally about a ``boundary'', which should be the points that are close to both the set and its boundary.
\begin{definition}[Interior, exterior, closure, boundary]
    Let $A \subseteq X$.
    \begin{enum}
        \item The \emph{interior $\interior(A)$ of $A$} is the unique open set satisfying 
        \begin{enum2}
            \item $\interior(A) \subseteq A$; and
            \item if $O \subseteq A$ is open, $O \subseteq \interior(A)$.
        \end{enum2}
        \item The \emph{exterior $\ext(A)$ of $A$} is the interior of $A^c$.
        \item The \emph{closure $\ol{A}$ of $A$} is the unique closed set satisfying
        \begin{enum2}
            \item $A \subseteq \ol{A}$; and
            \item if $C \supseteq A$ is closed, then $C \supseteq \ol{A}$.
        \end{enum2}
        \item The \emph{boundary $\partial A$ of $A$} is $\partial A \defeq \ol{A} \cap \ol{A^c}$.
    \end{enum}
\end{definition}
The definitions of the interior and closure of $A$ correspond to saying that $\interior(A)$ is the largest open set contained in $A$, and that $\ol A$ is the smallest closed set containing $A$. The following proposition makes this precise, alongside giving a way in which the interior and closure of a set are ``dual''.
\begin{proposition}
    Let $A \subseteq X$. Then
    \begin{enum}
        \item $$
            \interior(A) = \bigcup_{O \subseteq A, O \in \Tau} O
        $$
        \item $$
            \ol{A} = \bigcap_{C \supseteq A, C^c \in \Tau} C
        $$
        \item $\interior(A) = (\ol{A^c})^c$, or equivalently $\ol{A} = \interior(A^c)^c = \ext(A)^c$
        \item $\partial A = \partial A^c$
    \end{enum}
\end{proposition}
\begin{proof}
    To show (1), we see that the forward inclusion follows as $\interior(A) \subseteq A$ is a term in the union, and the latter follows as each individual $O \subseteq A$ is contained in $\interior(A)$. Similarly, in $(2)$, the forward inclusion follows as every $C \supseteq A$ has $C \supseteq \ol{A}$, and the reverse inclusion comes as $\ol{A} \supseteq A$ is a term in the union.

    We give a slightly more conceptual proof, and then a proof based on (1) and (2). Note that $\interior(A)^c \supseteq A^c$ is closed, and so $\interior(A)^c \supseteq \ol{A^c}$ as $\ol{A^c}$ is the minimal closed set containing $A^c$, and thus $\interior(A) \subseteq (\ol{A^c})^c$. Similarly, $(\ol{A^c})^c \subseteq A$ is open, so $(\ol{A^c})^c \subseteq \interior(A)$ since $\interior(A)$ is the maximal open set contained in $A$.

    Alternatively, we can compute
    $$
        (\ol{A^c})^c = \brac{\bigcap_{C \supseteq A^c, C^c \in \Tau} C}^c = \bigcup_{C^c \subseteq A, C^c \in \Tau}C^c = \interior(A)
    $$
    by reindexing $O \defeq C^c$. (4) follows as $\partial A^c = \ol{A^c} \cap \ol{(A^c)^c} = \ol{A} \cap \ol{A^c} = \partial A$.
\end{proof}
\subsection{Bases}
In other (algebraic) objects such as vector spaces and ideals in rings, it is usually equivalent and easier to consider a generating subset. In these cases we tend to prefer minimal generating subsets (such as a basis for a vector space), as a smaller generating subset tends to be more illuminating to the structure of the associated object. We have the following analogous notion in a topological space.
\begin{definition}[Basis]
    A \emph{basis (or base) for $X$} is a subset $\B \subseteq \Tau$ with the property that for every $U \in \Tau$, there is $\B_U \subseteq \B$ such that
    $$
        U = \bigcup_{B \in \B_U}B
    $$
\end{definition}
In this sense a basis $\B$ is a ``spanning set'' for $\Tau$ (in the sense that every $U \in \Tau$ can be reached as a union of $B \in \B$). In comparison to vector spaces we don't have a sensible notion of ``independence'' for open sets, nor is there a sensible notion of a minimal subset of $\Tau$ with the above properties in general.

The following examples illustrate the notion of a basis.
\begin{example}
    \begin{enum2}
    \item $\Tau$ is a basis for $X$, but this does not give any additional insight into the structure of the topological space.
    \item For $X = \R^n$ with the usual (metric) topology, we have a countable basis given by balls of rational radius with rational centre. That is,
    $$
        \B_{\R^n} = \set{B_r(x) \mid r \in \Q, x \in \Q^n}
    $$
    In fact, for any dense $E \subseteq \R$, the collection of $B_r(x)$ for $r \in E$ and $x \in E^n$ is a basis for $\R^n$ as a topological space.
    \end{enum2}
\end{example}
We should expect that a basis for a topological space cannot be ``much smaller'' than the space itself. The following result gives an explicit bound on the size of a basis relative to the size of the ambient space, and shows that the above bases for $\R^n$ are of minimal cardinality.
\begin{proposition}
    Let $\B$ be a basis for $X$. Then $\abs{\Tau} \leq \abs{\P(\B)}$.
\end{proposition}
\begin{proof}
    Let $\B \subseteq \Tau$ be a basis. Then
    $$
        \Tau = \set{\bigcup_{B \in \B_i}B \ \bigg| \ \B_i \subseteq \B}
    $$
    as each set in $\Tau$ is a union of sets in $\B$. The latter set is indexed by $\P(\B)$, and so the map
    \begin{align*}
        \P(\B) &\to \Tau \\
        \B_i &\mapsto \bigcup_{B \in \B_i}B
    \end{align*}
    is surjective, and taking cardinalities yields $\abs{\Tau} \leq \abs{\P(\B)}$.
\end{proof}
We have an equivalent and sometimes more useful formulation of what it means to be a basis. In essence, the idea of this statement comes from the fact that $U = \bigcup_{i \in I} U_i$ if and only if every $x \in U$ has $x \in U_i \subseteq U$.
\begin{lemma}
    A subset $\B \subseteq \Tau$ is a basis for $X$ if and only if for every $U \in \Tau$ and $x \in U$, there is $B_x \in \B$ with $x \in B_{x, U} \subseteq U$.
\end{lemma}
\begin{proof}
    If $\B \subseteq \Tau$ is a basis for $X$, each $U = \bigcup_{B \in \B_U}B$ for some $\B_U \subseteq \B$, and so each $x \in U$ has some $B$ (which we call $B_{x, U}$) so that $x \in B_{x, U} \subseteq \B_u$. Alternatively, if $\B$ has the latter property, then each $U = \bigcup_{x \in U} B_{x, U}$.
\end{proof}
If $X$ is a set, we may also ask what collections of subsets of $X$ form bases for topologies, and this is answered exactly in the following result.
\begin{proposition}
    A subset $\B \subseteq \P(X)$ is a basis for a (unique) topology on $X$ if and only if
    \begin{enum}
        \item $\bigcup_{B \in \B}B = X$; and
        \item For all $B_1, B_2 \in \Tau$ and $x \in B_1 \cap B_2$, there is $B_3$ with $x \in B_3 \subseteq B_1 \cap B_2$.
    \end{enum}
\end{proposition}
\begin{proof}
    If $\B$ is a basis for some topology $\Tau$, then $X$ is a union of some sets in $\B$, so 
    $$
        X = \bigcup_{B \in \B_i}B \subseteq \bigcup_{B \in \B}B
    $$
    for some $\B_i \subseteq \B$ and each $B \subseteq X$, so $\bigcup_{B \in \B_i}B \subseteq X$, showing this equality. If $B_1, B_2 \in \B$, then $B_1 \cap B_2 = \bigcup_{B \in B_i}B$ for some $\B_i \subseteq \B$, and so each $x \in B_1 \cap B_2$ is in some $B_3 \in \B_i \subseteq \B$.

    Now suppose that $\B$ satisfies the two given constraints. Then
    \begin{enum}
        \item $\bigcup_{B \in \emptyset} B = \emptyset$ and $\bigcup_{B \in \B} B = X$;
        \item If $B_i = \bigcup_{B \in \B_i} B$ for a collection of subsets $\set{B_i \mid i \in I}$, then $\bigcup_{i \in I} \B_i \subseteq \B$, and
        $$
            \bigcup_{i \in I} B_i = \bigcup_{B \in \bigcup_{i \in I} \B_i} B
        $$
        \item It suffices to show that $B_1 \cap B_2$ is a union of sets from $\B$. For each $x \in B_1 \cap B_2$ there is $B_x \in \B$ with $x \in B_x \subseteq B_1 \cap B_2$, and so writing
        $$
            \B_{12} = \set{B_x \mid x \in B_1 \cap B_2} \subseteq \B
        $$
        we have
        $$
            B_1 \cap B_2 = \bigcup_{x \in B_1 \cap B_2} B_x = \bigcup_{B \in \B_{12}} B
        $$
    \end{enum}
\end{proof}


% Basis reformulation
% Topology basis lemma

% Subbases

% Open sets are ``large'' and closed sets are ``small''

% Open sets can be viewed (from internally) as ``unbounded'', while closed sets have some notion of ``boundary''.

% Bases and ``spanning''

% Equivalent formulation for a basis

% Density

\subsection{Continuous maps}

% Continuity from a metric and algebraic perspective
% Classical metric space formulation

% ``Arrows'' i.e. morphisms
% Pullbacks of open sets, compatibilities of topologies

% Equivalent formulations

% Homeomorphisms specifying that topological properties are fully preserved
% Remark about languages

% Bases, subbases

\subsection{Separability}
% Separable, first and second countability, T1 through to T4 spaces.

\subsection{Induced topologies}

% Subspaces, products, quotients (universal properties)

\subsection{Compactness}

\subsection{Connectedness}

% Connectedness, compactness

\subsection{Locality}
% Local variations of properties

% Sequences and related discussion
%   In abstract metric space theory, sequences play an important role and are on the same level as many non-sequence constructions.

% Algebraic perspectives?
\end{document}